\section{Abstract}
Published in 1990, John Hughes' article on {\it Why Functional Programming Matters} highlighted its importance for coping with the demands of structuring ever more complex software\cite{WhyFunctional}. With the advent of the internet and processors with multiple cores, the functional paradigm has only grown in value for its ability to concisely express code which is asynchronous or concurrent code - that is to say fetches data over a network or is run in parallel on multiple processors.

The goal of this paper is to explore the fields of mathematical theory which give functional programming the power to do this. Our journey begins by examining the first mathematical model of a computer, the Turing Machine, before stripping the notion of computation down to its bare essence and building it back up with models of logic pioneered in the 1920s and 30s by logicians Schönfinkel, Curry, and Church\cite{LambdaAndCombinatorsIntro}.

These models of logic led to type systems which, bolstered by category theory, gave rise to the monad, an extremely versatile computational interface. Among the monad's uses, one can include the parsing of text into data structures but we shall focus on its application in asynchronous programming.